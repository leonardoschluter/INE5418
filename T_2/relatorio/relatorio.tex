\documentclass{article}
\usepackage[utf8]{inputenc}
\usepackage{listings}
\usepackage{float}
\usepackage{graphicx}
\PassOptionsToPackage{hyphens}{url}\usepackage{hyperref}

\title{Trabalho 2 de Computação Distribuída - INE5418}
\author{Leonardo Schlüter \\ 13200658}
\date{03 de maio, 2021}

\usepackage{indentfirst}

\begin{document}

\maketitle

\section{Ideia de Aplicação e Requisitos}

A aplicação idealizada para esse projeto é algo semelhante ao \href{https://gartic.com.br/}{jogo gartic}. Os requisitão são então, numa versão mais básica, uma plataforma de desenho visível para todos os jogadores, sendo que apenas um jogador pode desenhar por rodada. Um chat verificador de respostas dos outros jogadores que estão apenas observando o desenho. Sobre o chat, vale ressaltar que a pontuação será baseado numa ordem FIFO, então as respostas dos jogadores serão infileiradas e pontuadas baseado num valor decrescente, começando em X e indo até 1, sendo X os números de jogadores menos o desenhista, o nodo coordenador do chat pode ser justamente o do jogador que está desenhando. Um coordenador (Máquina) que escolha palavras para serem desenhadas, escolha jogadores para desenharem e mantenha os pontos de cada jogador. 


\section{MVP da implementação}

Sendo um grande fã de NBA, resolvi fazer o trocadilho ( Most Valuable Player - Most Valuable Part of the code ). Na implementação utilizei de base as Demos de Chat e Draw para ter uma base para o jogo. Mas muitos desafios surgem baseado nos requisitos inicias dados acima:


\subsection{Desafio: deixar o demo Draw desenhável apenas por um usuário}

O desafio é autoexplicativo, atualmente o Demo abre uma instância do panel para todos e permite que todos interajam com o mesmo. Para arrumar isso foi alterado o seguinte: 


\begin{lstlisting} 
asm!(
    "csrw mtvec, {}",
    in(reg) (mtvec_clint_vector_table as usize | 0x1)
);

asm!("csrw mstatus, 0b1 << 3");

asm!("csrw mie, 0b1 << 3");
\end{lstlisting}
    
\section{Referências}

\begin{enumerate}
 \item SiFive Interrupt cookbook - Disponível em \url{https://sifive.cdn.prismic.io/sifive/d1984d2b-c9b9-4c91-8de0-d68a5e64fa0f_sifive-interrupt-cookbook-v1p2.pdf}
 \item An Introduction to RISC-V Architecture - Disponível em \url{https://cdn2.hubspot.net/hubfs/3020607/An%20Introduction%20to%20the%20RISC-V%20Architecture.pdf}
 \item The RISC-V Instruction Set Manual Volume II: Privileged Architecture Privileged Architecture Version 1.10 - Disponível em \url{https://riscv.org//wp-content/uploads/2017/05/riscv-privileged-v1.10.pdf}
 \item SiFive FU540-C000 Manual - Disponível em \url{https://sifive.cdn.prismic.io/sifive%2F834354f0-08e6-423c-bf1f-0cb58ef14061_fu540-c000-v1.0.pdf}
\end{enumerate}

\end{document}